%% The first command in your LaTeX source must be the \documentclass command.
\documentclass[sigconf, anonymous]{acmart}

\def\BibTeX{\textsc{Bib}\TeX}
\def\LaTeX{\textsc{La}\TeX}
\usepackage{url}
\usepackage{balance}
\usepackage{color}
\usepackage{enumerate}
\settopmatter{printacmref=false}

\setcopyright{none}
\renewcommand\footnotetextcopyrightpermission[1]{}
\pagestyle{plain}
\acmConference[CSCI]{}{Term Paper}{2021}

\begin{document}
\title{Research Paper (Rename Appropriately)}

\author{Anonymous}
\affiliation{
  \institution{}
}

\begin{abstract}
  The abstract should be two paragraphs that summarize your research
  paper: the first setting up the background and the second
  summarizing what learned from the research.

  Abstracts are read independently from the rest of the paper so you
  cannot cite any other papers in it. Study other abstracts in the
  papers you are reading to understand what an abstract should really
  means. {\bf You must write the abstract in past tense and in third
    person.}

  The abstract must make it clear what your report is about: the work
  that you did for your research, and it should be written up as a
  research report. Make sure you take credit for what you and briefly
  mention the work (research ideas, software, etc.) done by others.

  The abstract is not an introduction or overview of your paper! It is
  a summary of your paper, which should include the background,
  context, content, and contributions of your report. It should
  typically be around 150-200 words.
\end{abstract}

\keywords{Come up with your own keywords.}

\maketitle

\section{Introduction}
\label{intro}

Use this section to provide a short presentation of the research
problem that your selected papers discuss, and why this problem(s) is
important.

(For Phase 1--Proposal, write down the needed paragraph in this
section. For the other sections, leave the section headers in place
but delete the text within the sections. Of course, you must use
BibTeX to generate the References section for your selected papers.)

Eventually provide the roadmap for the remaining sections of the
paper. For example, you can state that Section~\ref{related} discusses
the related work in the overall area of the problem and
section~\ref{research problem} describes the research problem in
detail, along with the proposed soultions. Section~\ref{analysis}
discusses what you learned about solving the research problem, its
current status and future work that is needed in the areas.
Similarly, mention the other sections in your paper using appopriate
references.

\section{Related Work}
\label{related}

Use this section to present the literature survey of all the papers
that focus on the selected research problem and others that address
simlar problems. Show that you have a broad understanding of this
research space.

\section{Research Problem}
\label{research problem}

Use this section to describe the research problem and then present the
different solution approaches in the selected papers. What do the
authors agree are the common themes for this topic?

\section{Analysis}
\label{analysis}

Use this section to analyze the rsearch problems and the different
solutions that have been proposed; where do these methods fail? How
well did the authors present and explain their work, and how important
is the work to the body of knowledge? Also, discuss the current status
and future prospects for this area.

\section{Legal Considerations}
\label{legal considerations}

Use this section to discuss legal issues relevant to your topic.

Use your general readings to guide the legal aspects of your
discussion. Look at the laws that have been passed in recent years,
and look at legislation that is being proposed in the space covered by
your topic. Cite any significant laws, i.e., you need to have them in
the list of references for them.

\section{Ethical Considerations}
\label{ethical considerations}

Use this section to discuss ethical issues relevant to your topic.

Use the ACM Code to guide the ethical aspects of your
discussion here~\cite{ACMCODE}.

Note. This section is not necessarily about what the papers cover, but
about what legal and ethical considerations apply to the data and
discussions that come out of the paper.

\section{Conclusions}
\label{conclusions}

Use this section to summarize your conclusions. Describe what {\bf
  you} now concluded about these selected papers based on your
understanding.

You should also discuss possible future research directions for the
selected topic.


%%%%%%%%%%%%%%%%%%%%%%%%%%%%%%%%%%%%%%%%%%%%%%%%%
%%% DELETE FROM HERE 
%%%%%%%%%%%%%%%%%%%%%%%%%%%%%%%%%%%%%%%%%%%%%%%%%

\section*{Tables, Figures, and Citations/References -
  DELETE THIS SUBSECTION BEFORE ANY SUBMISSION}

{\bf This unnumbered section is meant to provide you with some help in
  dealing with figures, tables and citations, as these are sometimes
  hard for people new to \LaTeX. Your figures, tables and citations
  must be distributed all over your paper (not here), as appropriate
  for your paper. So here is a quick guide extracted from the ACM
  style guide.

  Please delete this entire section before you make any submission! If
  I see this section in your report, you will lose points!!!}

\begin{table}
\centering
\caption{Issue Resolution}
\label{SAMPLE TABLE}
\begin{tabular}{|l|r|l|} \hline
Issue&Percentage&Summary\\ \hline
Issue 1 &  5\% & Assign the best programmers\\ \hline
Issue 2 &  30\% & Assign the new full-time hires\\ \hline
Issue 3 &  70\% & Assign the new co-op students on this\\ \hline
Issue 4 &  90\% & Can be kept on the back-burner for now\\ \hline
\end{tabular}
\end{table}

Tables, figures, and citations/references in technical documents need
to be presented correctly. In proper technical English writing (for
reasons beyond the scope of this discussion), table captions are above
the table and figure captions are below the figure. So the issue
resultion status of this nonsensical project is shown in
Table~\ref{SAMPLE TABLE}. Note that tables are never above or below,
as the typesetting is at liberty to place them anywhere meaningful

Note that figures in the research paper must be original, that is, created
by the student: please do not screen-scrape and cut-and-paste figures
from any other paper you have read. Just cite the figure in the paper
and summarize what points you want to make.

When you need to cite any original figures in your own paper, they
should be handled as demonstrated here. State that Figure~\ref{SAMPLE
  FIGURE} is a simple illustration used in the ACM Style sample
document. Again, never refer to the figure below (or above) because
figures may be placed by \LaTeX{} at any appropriate location that can
change when you recompile your source $.tex$ file.

\begin{figure}[htb]
\begin{center}
\includegraphics[width=1.5in]{rit-tiger-with-text.jpg}
\caption{The cutest tiger in the world (JPG).}
\label{SAMPLE FIGURE}
\end{center}
\end{figure}

Finally, citing documents needs to be done properly too. For example,
a paper by Mic Bowman, Saumya K. Debray, and Larry L. Peterson could
be cited as Bowman, Debray, and Peterson~\cite{bowman:reasoning}. A
set of papers could collectively be cited as the literature in this
area consists of several interesting
papers~\cite{braams:babel,clark:pct,herlihy:methodology}. One of the
common types of citations these days is to items only posted on the
Web such as this 2014 CMU SEI webinar by Dormann et al.~\cite{dormann:API}.

You will find the \BibTeX{} entries needed for many papers that are
being cited at the ACM or IEEE digital libraries, or other sources on
the web, otherwise you can write your own versions easily and add them
to the $*.bib$ file in the folder. There are many sample bibtex
template files that can be used to model your own references.

The list of all references will be generated in the standard ACM Reference
style using \LaTeX{}/\BibTeX{} correctly. Note that you
need to first the following sequence to get the paper
compiled correctly:

\begin{enumerate}
\item {\tt latex} {\em researchpaper}
\item {\tt bibtex} {\em researchpaper}
\item {\tt latex} {\em researchpaper}
\item {\tt latex} {\em researchpaper}
\end{enumerate}

%%%%%%%%%%%%%%%%%%%%%%%%%%%%%%%%%%%%%%%%%%%%%%%%%
%%% DELETE UNTIL HERE
%%%%%%%%%%%%%%%%%%%%%%%%%%%%%%%%%%%%%%%%%%%%%%%%%


\balance
\bibliographystyle{ACM-Reference-Format}
\bibliography{researchpaper} 

\end{document}
